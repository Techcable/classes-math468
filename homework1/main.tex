\documentclass[14pt,letterpaper]{extarticle}
\usepackage[utf8]{inputenc}

% Set margins
% https://tex.stackexchange.com/a/327136
% this should replace use of the 'fullpage' package
\usepackage[
    includehead,
    margin=2.5cm,
    bmargin=2cm,
]{geometry}

\usepackage{amssymb}
\usepackage{amsthm}
\usepackage{verbatim} % This works better than comment for some reason
\usepackage{hyperref}
\usepackage{amsmath}
\usepackage{gensymb}
\usepackage{enumitem}
\usepackage{xfrac} % Fractions - https://tex.stackexchange.com/a/230973
\usepackage{xcolor}
% The magic latex drawing thing
\usepackage{tikz}
\usetikzlibrary{shapes}
\usepackage[en-US,showdow]{datetime2}

% Colored hyperlinks:
% https://www.overleaf.com/learn/latex/Hyperlinks
\hypersetup{
    colorlinks=true,
    urlcolor=blue
}

% Setup nice headings
% https://www.overleaf.com/learn/latex/Headers_and_footers
\usepackage{fancyhdr}
\usepackage{titling} % Save title for use in fancyhdr

% Shortcuts
\newcommand*{\bP}{\mathbf{P}}



% Use black tombstone for QED
\renewcommand{\qedsymbol}{$\blacksquare$}

% Important sets
\newcommand*{\C}{\mathbb{C}}
\newcommand*{\R}{\mathbb{R}}
\newcommand*{\Q}{\mathbb{Q}}
\newcommand*{\Z}{\mathbb{Z}}

%
% symbols
%

% name by meaning, not by character
\newcommand{\grad}{\nabla}


% New command for 'partial QED' using white tombstone
\newcommand*{\partialqed}{$\hfill \square$ \\}

\newcommand*{\todo}{{\color{red}\textbf{TODO:}} }

% Increased spaces spacing
\usepackage{setspace} \onehalfspacing

\title{Math 468 - Homework 1}
\author{Nicholas Schlabach (nickninja@arizona.edu)}
\DTMsavedate{titledate}{2026-01-24}
\date{\DTMUsedate{titledate}}

%% Copied from https://tex.stackexchange.com/a/2238
%% Modified by Techcable 9/19/21 to use '[...]' instead of '(...)'
\newenvironment{amatrix}[1]{%
  \left[\begin{array}{@{}*{#1}{c}|c@{}}
}{%
  \end{array}\right]
}

\begin{document}
\pagestyle{fancy}
\fancyhead[L]{Nicholas Schlabach}
\fancyhead[C]{\hspace{5mm}\Big(\thetitle{}\Big)}
\fancyhead[R]{{\DTMsetdatestyle{mmddyy}\DTMusedate{titledate}}, Page \thepage}
\fancyfoot{} % no footer
\maketitle
% maketitle automatically sets pagestyle=plain, override this
\thispagestyle{fancy}


\begin{center}
\begin{enumerate}
\item

        Squaring $\mathbf{P}$ gives the following matrix:
        \[ \mathbf{P}^2=\begin{bmatrix}
            p^2 + p (q - 2) + 1 & -p (p + q - 2) \\
-q (p + q - 2) & (p - 2) q + q^2 + 1)
            \end{bmatrix}
        \]
    \begin{enumerate}
    \item
        By theorem, we know $\pi_2=\pi_0 \mathbf{P}^2$. Since we are computing the conditional probability on the assumption $X_0=s_0$, then we want to consider the case that $\pi_0=\begin{bmatrix}
            1 & 0
        \end{bmatrix}$. Matrix multiplying shows that $\pi_2(0)$ simplifies to the upper-left entry of the matrix, giving a final answer of $P(X_2=s_0|X_0=s_0)=p^2+p(q-2)+1$.
    \item
        We can evaluate $\pi_2(0)$ by matrix multiplication. We only need to partially multiply as we only care about the first entry:
        \[ \pi_2(0)=\frac{1}{3}\Big(p^2+p(q-2)+1\Big)+\frac{2}{3}\Big(-q(p+q-2)\Big) \]
        This value is the desired probability $P(X_2=s_1)$.
    \item
        Let $A, B, C$ denote $X_0=s_0,X_1=s_0,X_2=2_0$ respectively, so that the desired quantity is $P(B|A,C)$.  By the definition of conditional probability:
        \[ P(B|A,C)=\frac{P(X_0=s_0,X_1=s_0,X_2=s_0)}{P(X_0=s_0,X_2=s_0)} \]
        We can use the result of (a) to get $P(A,C)=P(A)P(C|A)=\pi_0(0)(p^2+p(q-2)+1)$. Compmuting the probability of the intersection gives 
        \[ P(A,B,C)=P(B,C|A)P(A)=P(C|B,A)P(B|A)P(A)  \]
        Using the markov property, $P(C|B,A)=P(C|B)$. This means that the probabilities can be computed using a transition matrix and $\pi_0$. In particular $P(C|B)=1-p$, $P(B|A)=1-p$. This means that $P(A,B,C)=(1-p)^2\pi_0(0)$.
        Combining with the previous result gives  the final answer:
    \[ P(B|A,C)=\frac{(1-p)^2}{p^2+p(q-2)+1} \]

\end{enumerate}
\item
    We want to consider fromula (29) on page 15 in the two state case $S=\{0,1\}$ for arbitrary $x,y$.  In the two-state case, the sum $\sum_{z\ne y}$ only has a single case where $z=1-y$.
    This means the formula becomes
        \begin{align}
            P_x(T_y=n+1)&=\bP(x,1-y)P_{1-x}(T_y=n)  & n \ge 1
        \end{align}
    This is kind of like a recurrence relation, but a little different because we alternate between $P_x(T_y=\dots)$ and $P_{x-1}(T_y=\dots)$. Applying (1) to itself elimnates this problem, creating a true recurrence relation:
    \begin{align}
    P_x(T_y=n+2)&=\bP(x,1-y)\bP(1-x,1-y)P_x(T_y=n) & n \ge 1  
    \end{align}
    \begin{enumerate}
    \item
        We want to consider equation (2) where $y=0,x=0$. The values in (2) then become $\bP(x,1-y)=\bP(0,1)=p$ and $\bP(1-x,1-y)=\bP(1,1)=1-q$:
        \begin{align}
            \label{problem2generic}
             P_x(T_y=n+2)&=p(1-q)P(T_y=n)
        \end{align}
        Notice that this equation is also valid for $x=1$, as the product is merly swapped and multiplication is commutative. This means \eqref{problem2generic} is valid for any $x\in S$, as long as $y=0$; \\
        Split into cases depending on $n\ge 1$: \\
        Case 1: If $n=2k+1$ is odd. \\
        In this case, we apply (2) exactly $k$ times, going from $n=1$ to $n=2k+1$. This gives the result 
        \[ P_x(T_y=n)=p^k(1-q)^k(1-p) \]
        This equation is still valid where $n=1$ and $k=0$. \\
        Case 2: If $n=2k$ even and $n\ge 2$. \\
        Apply the result of case 1 to $n'=2k-1\ge 1$. This gives $P_x(T_y=2k-1)=p^k(1-q)^k(1-p)$. Apply equation (1) to get 
        \[ P_x \]
    \end{enumerate}
\end{enumerate}
\end{center}
\end{document}