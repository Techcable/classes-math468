\documentclass[14pt,letterpaper]{extarticle}
\usepackage[utf8]{inputenc}

% Set margins
% https://tex.stackexchange.com/a/327136
% this should replace use of the 'fullpage' package
\usepackage[
    includehead,
    margin=2.5cm,
    bmargin=2cm,
]{geometry}

\usepackage{amssymb}
\usepackage{amsthm}
\usepackage{verbatim} % This works better than comment for some reason
\usepackage{hyperref}
\usepackage{amsmath}
\usepackage{gensymb}
\usepackage{enumitem}
\usepackage{xfrac} % Fractions - https://tex.stackexchange.com/a/230973
\usepackage{xcolor}
% The magic latex drawing thing
\usepackage{tikz}
\usetikzlibrary{shapes}
\usepackage[en-US,showdow]{datetime2}

% Colored hyperlinks:
% https://www.overleaf.com/learn/latex/Hyperlinks
\hypersetup{
    colorlinks=true,
    urlcolor=blue
}

% Setup nice headings
% https://www.overleaf.com/learn/latex/Headers_and_footers
\usepackage{fancyhdr}
\usepackage{titling} % Save title for use in fancyhdr


% Use black tombstone for QED
\renewcommand{\qedsymbol}{$\blacksquare$}

% Important sets
\newcommand*{\C}{\mathbb{C}}
\newcommand*{\R}{\mathbb{R}}
\newcommand*{\Q}{\mathbb{Q}}
\newcommand*{\Z}{\mathbb{Z}}

%
% symbols
%

% name by meaning, not by character
\newcommand{\grad}{\nabla}


% New command for 'partial QED' using white tombstone
\newcommand*{\partialqed}{$\hfill \square$ \\}

\newcommand*{\todo}{{\color{red}\textbf{TODO:}} }

% Increased spaces spacing
\usepackage{setspace} \onehalfspacing

\title{Math 468 - Homework 1}
\author{Nicholas Schlabach (nickninja@arizona.edu)}
\DTMsavedate{titledate}{2026-01-24}
\date{\DTMUsedate{titledate}}

%% Copied from https://tex.stackexchange.com/a/2238
%% Modified by Techcable 9/19/21 to use '[...]' instead of '(...)'
\newenvironment{amatrix}[1]{%
  \left[\begin{array}{@{}*{#1}{c}|c@{}}
}{%
  \end{array}\right]
}

\begin{document}
\pagestyle{fancy}
\fancyhead[L]{Nicholas Schlabach}
\fancyhead[C]{\hspace{5mm}\Big(\thetitle{}\Big)}
\fancyhead[R]{{\DTMsetdatestyle{mmddyy}\DTMusedate{titledate}}, Page \thepage}
\fancyfoot{} % no footer
\maketitle
% maketitle automatically sets pagestyle=plain, override this
\thispagestyle{fancy}


\begin{center}
\begin{enumerate}
\item

        Squaring $P$ gives the following matrix:
        \[ P^2=\begin{bmatrix}
            p^2 + p (q - 2) + 1 & -p (p + q - 2) \\
-q (p + q - 2) & (p - 2) q + q^2 + 1)
            \end{bmatrix}
        \]
    \begin{enumerate}
    \item
        By theorem, we know $\pi_2=\pi_0 P^2$. Since we are computing the conditional probability on the assumption $X_0=s_0$, then we want to consider the case that $\pi_0=\begin{bmatrix}
            1 & 0
        \end{bmatrix}$. Matrix multiplying shows that $\pi_2(0)$ simplifies to the upper-left entry of the matrix, giving a final answer of $P(X_2=s_0|X_0=s_0)=p^2+p(q-2)+1$.
    \item
        We can evaluate $\pi_2(0)$ by matrix multiplication. We only need to partially multiplyas we only care about the first entry:
        \[ \pi_2(0)=\frac{1}{3}\Big(p^2+p(q-2)+1\Big)+\frac{2}{3}\Big(-q(p+q-2)\Big) \]
        This value is the desired probability $P(X_2=s_1)$.
\end{enumerate}
\end{enumerate}
\end{center}
\end{document}