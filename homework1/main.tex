\documentclass[14pt,letterpaper]{extarticle}
\usepackage[utf8]{inputenc}

% Set margins
% https://tex.stackexchange.com/a/327136
% this should replace use of the 'fullpage' package
\usepackage[
    includehead,
    margin=2.5cm,
    bmargin=2cm,
]{geometry}

\usepackage{amssymb}
\usepackage{amsthm}
\usepackage{verbatim} % This works better than comment for some reason
\usepackage{hyperref}
\usepackage{amsmath}
\usepackage{gensymb}
\usepackage{enumitem}
\usepackage{xfrac} % Fractions - https://tex.stackexchange.com/a/230973
\usepackage{xcolor}
% The magic latex drawing thing
\usepackage{tikz}
\usetikzlibrary{shapes}
\usepackage[en-US,showdow]{datetime2}

% Colored hyperlinks:
% https://www.overleaf.com/learn/latex/Hyperlinks
\hypersetup{
    colorlinks=true,
    urlcolor=blue
}

% Setup nice headings
% https://www.overleaf.com/learn/latex/Headers_and_footers
\usepackage{fancyhdr}
\usepackage{titling} % Save title for use in fancyhdr

% Shortcuts
\newcommand*{\bP}{\mathbf{P}}



% Use black tombstone for QED
\renewcommand{\qedsymbol}{$\blacksquare$}

% Important sets
\newcommand*{\C}{\mathbb{C}}
\newcommand*{\R}{\mathbb{R}}
\newcommand*{\Q}{\mathbb{Q}}
\newcommand*{\Z}{\mathbb{Z}}

%
% symbols
%

% name by meaning, not by character
\newcommand{\grad}{\nabla}


% New command for 'partial QED' using white tombstone
\newcommand*{\partialqed}{$\hfill \square$ \\}

\newcommand*{\todo}{{\color{red}\textbf{TODO:}} }

% Increased spaces spacing
\usepackage{setspace} \onehalfspacing

\title{Math 468 - Homework 1}
\author{Nicholas Schlabach (nickninja@arizona.edu)}
\DTMsavedate{titledate}{2026-01-24}
\date{\DTMUsedate{titledate}}

%% Copied from https://tex.stackexchange.com/a/2238
%% Modified by Techcable 9/19/21 to use '[...]' instead of '(...)'
\newenvironment{amatrix}[1]{%
  \left[\begin{array}{@{}*{#1}{c}|c@{}}
}{%
  \end{array}\right]
}

\begin{document}
\pagestyle{fancy}
\fancyhead[L]{Nicholas Schlabach}
\fancyhead[C]{\hspace{5mm}\Big(\thetitle{}\Big)}
\fancyhead[R]{{\DTMsetdatestyle{mmddyy}\DTMusedate{titledate}}, Page \thepage}
\fancyfoot{} % no footer
\maketitle
% maketitle automatically sets pagestyle=plain, override this
\thispagestyle{fancy}


\begin{center}
\begin{enumerate}
\item

        Squaring $\mathbf{P}$ gives the following matrix:
        \[ \mathbf{P}^2=\begin{bmatrix}
            p^2 + p (q - 2) + 1 & -p (p + q - 2) \\
-q (p + q - 2) & (p - 2) q + q^2 + 1)
            \end{bmatrix}
        \]
    \begin{enumerate}
    \item
        By theorem, we know $\pi_2=\pi_0 \mathbf{P}^2$. Since we are computing the conditional probability on the assumption $X_0=s_0$, then we want to consider the case that $\pi_0=\begin{bmatrix}
            1 & 0
        \end{bmatrix}$. Matrix multiplying shows that $\pi_2(0)$ simplifies to the upper-left entry of the matrix, giving a final answer of $P(X_2=s_0|X_0=s_0)=p^2+p(q-2)+1$.
    \item
        We can evaluate $\pi_2(0)$ by matrix multiplication. We only need to partially multiply as we only care about the first entry:
        \[ \pi_2(0)=\frac{1}{3}\Big(p^2+p(q-2)+1\Big)+\frac{2}{3}\Big(-q(p+q-2)\Big) \]
        This value is the desired probability $P(X_2=s_1)$.
    \item
        Let $A, B, C$ denote $X_0=s_0,X_1=s_0,X_2=2_0$ respectively, so that the desired quantity is $P(B|A,C)$.  By the definition of conditional probability:
        \[ P(B|A,C)=\frac{P(X_0=s_0,X_1=s_0,X_2=s_0)}{P(X_0=s_0,X_2=s_0)} \]
        We can use the result of (a) to get $P(A,C)=P(A)P(C|A)=\pi_0(0)(p^2+p(q-2)+1)$. Compmuting the probability of the intersection gives 
        \[ P(A,B,C)=P(B,C|A)P(A)=P(C|B,A)P(B|A)P(A)  \]
        Using the markov property, $P(C|B,A)=P(C|B)$. This means that the probabilities can be computed using a transition matrix and $\pi_0$. In particular $P(C|B)=1-p$, $P(B|A)=1-p$. This means that $P(A,B,C)=(1-p)^2\pi_0(0)$.
        Combining with the previous result gives  the final answer:
    \[ P(B|A,C)=\frac{(1-p)^2}{p^2+p(q-2)+1} \]

\end{enumerate}
\item
    \begin{enumerate}
    \item

        We are interested in a probability of starting at state $s_0$, avoiding it for $n-1$ steps and then first hitting it at step 1. The only way to avoid $s_0$ is by being at state $s_1$, meaning we immediately transition to $s_1$, stay there and transition back to $s_0$ in the final step. This has probability $\bP(0,1)\bP(1,1)^{n-2}\bP(1,0)$. There are no other possibilities to achieve a hitting time of $n$ when having only two states. This approach doesn't work if $n=1$, as we only have enough time to make a single step. Instead we just stay at state $s_0$, which has probability $\bP(0,0)$. Replacing the 
        \begin{align*}
                P_0(T_0=n)&=\bP(0,1)\bP(1,1)^{n-2}\bP(1,0)=p(1-q)^{n-2}q&  &n \ge 2 \\
                P_0(T_0=n)&=\bP(0,0)\hspace{37mm}=1-p & & n= 0
        \end{align*}
    \item
        The problem here is similar, but even simpler. We want to avoid $s_1$ until the final step $n$. Since we start at $s_0$, we just stay there for $n-1$ steps, then at the final step move from $s_0\to s_1$. Because the system is two state, there is no other way to achieve that hitting time. This works even for the $n=1$ case.
        \begin{align*}
             P_0(T_1=n) &=\bP(0,0)^{n-1}\bP(0,1)=(1-p)^{n-1}p & n\ge 1
        \end{align*}
    \end{enumerate}
\item
    \begin{enumerate}
    \item The states correspond to points on the unit circle with the following radians:
        \[ \{0,2\pi/N,2(2\pi/N),3(2\pi/N),\dots,(N-1)(2\pi/N)\} \] Call these points $s_0,\dots,s_{n-1}$ for convenience. The point $s_i$ has cartesian coordinaete $(\sin \big(i(2\pi/N)\big),\cos\big( i(2\pi/N)\big)$. In general, these positions are irrational. But for $N=6$ we get some well-known unit circle positions 
        \[ \{(0,1),(\frac{1}{2},\frac{\sqrt{3}}{2}),(-\frac{1}{2},\frac{\sqrt{3}}{2}),(-1,0),(-\frac{1}{2},-\frac{\sqrt{3}}{2}),(-\frac{1}{2},-\frac{\sqrt{3}}{2})\} \] 
        Moving counterclockwise, we go from $s_i$ to $s_{i+1}$ with probability $P$ whenever $0 \le i<N-1$. The other main movement is counterclockwise which goes from $s_{i+1}\to s_i$ with probability $1-p$. Adding in handling of the wrapparound, $\bP$ can be defined in the following way:
        \begin{align*}
            \bP(k,k+1)& =p & \text{if } k<N-1 \\
            \bP(k,k-1)& =1-p &  \text{if }k>0 \\
            \bP(N-1,0)&= p & \\
            \bP(0,N-1)&= 1-p & \\
            \bP(x,y)&=0 & \text{otherwise}
        \end{align*}
        When $N=6$ we get the following matrix
        \[ M=\begin{bmatrix}
            0 & p & 0 & 0 & 0 & 1-p \\
            1-p & 0 & p & 0 & 0 & 0 \\
            0 & 1-p & 0 & p & 0 & 0 \\
            0 & 0 & 1-p & 0 & p & 0  \\
            0 & 0 & 0 & 1-p & 0 & p  \\
            p & 0 & 0 & 0 & 1-p & 0  \\
        \end{bmatrix}\]
    \end{enumerate}
\end{enumerate}
\end{center}
\end{document}