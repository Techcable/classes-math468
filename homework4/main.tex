\documentclass[14pt,letterpaper]{extarticle}
\usepackage[utf8]{inputenc}

% Set margins
% https://tex.stackexchange.com/a/327136
% this should replace use of the 'fullpage' package
\usepackage[
    includehead,
    margin=2.5cm,
    bmargin=2cm,
]{geometry}

\usepackage{amssymb}
\usepackage{amsthm}
\usepackage{verbatim} % This works better than comment for some reason
\usepackage{hyperref}
\usepackage{amsmath}
\usepackage{gensymb}
\usepackage{enumitem}
\usepackage{xfrac} % Fractions - https://tex.stackexchange.com/a/230973
\usepackage{xcolor}
% The magic latex drawing thing

\usepackage{tikz}
\usetikzlibrary{shapes}
\usepackage[en-US,showdow]{datetime2}

% Colored hyperlinks:
% https://www.overleaf.com/learn/latex/Hyperlinks
\hypersetup{
    colorlinks=true,
    urlcolor=blue
}

% Setup nice headings
% https://www.overleaf.com/learn/latex/Headers_and_footers
\usepackage{fancyhdr}
\usepackage{titling} % Save title for use in fancyhdr

% Shortcuts
\newcommand*{\bP}{\mathbf{P}}



% Use black tombstone for QED
\renewcommand{\qedsymbol}{$\blacksquare$}

% Important sets
\newcommand*{\C}{\mathbb{C}}
\newcommand*{\R}{\mathbb{R}}
\newcommand*{\Q}{\mathbb{Q}}
\newcommand*{\Z}{\mathbb{Z}}

%
% symbols
%

% name by meaning, not by character
\newcommand{\grad}{\nabla}


% New command for 'partial QED' using white tombstone
\newcommand*{\partialqed}{$\hfill \square$ \\}

\newcommand*{\todo}{{\color{red}\textbf{TODO:}} }

% Increased spaces spacing
\usepackage{setspace} \onehalfspacing

\title{Math 468 - Homework 4}
\author{Nicholas Schlabach (nickninja@arizona.edu)}
\DTMsavedate{titledate}{2026-02-17}
\date{\DTMUsedate{titledate}}

%% Copied from https://tex.stackexchange.com/a/2238
%% Modified by Techcable 9/19/21 to use '[...]' instead of '(...)'
\newenvironment{amatrix}[1]{%
  \left[\begin{array}{@{}*{#1}{c}|c@{}}
}{%
  \end{array}\right]
}

\begin{document}
\pagestyle{fancy}
\fancyhead[L]{Nicholas Schlabach}
\fancyhead[C]{\hspace{5mm}\Big(\thetitle{}\Big)}
\fancyhead[R]{{\DTMsetdatestyle{mmddyy}\DTMusedate{titledate}}, Page \thepage}
\fancyfoot{} % no footer
\maketitle
% maketitle automatically sets pagestyle=plain, override this
\thispagestyle{fancy}


\begin{center}
\begin{enumerate}
\item
    \begin{comment}
    There are two ways to approach this problem.
    \vspace{3mm}\\
    Approach 1 (the way in the book):
    \end{comment}
    As demonstrated on page 49, the stationary distribution $\pi$ can be found by solving $\bP\pi=\pi$ subject to the condition $\sum_{i=0}^2 \pi_i=1$.
    \begin{align*}
         .4\pi_0+.4\pi_1+.2\pi_2 & =\pi_0 \\
         .3\pi_0+.4\pi_1+.3\pi_2 &=\pi_1 \\
         .2\pi_2+.4\pi_1+.4\pi_2&=\pi_2  \\
         \pi_0+\pi_1+\pi_2&=1
    \end{align*}
    This can be written as an augmented matrix 
    \[ \begin{amatrix}{3}
        -0.6 & .4 & .2 & 0 \\
        .3 & -0.6 & .3 & 0 \\
        .2 & .4 & -0.6 & 0 \\
        1 & 1 & 1 & 1
        
    \end{amatrix} \]
    Converting to reduced row-echelon form gives the following:
    \[ \begin{amatrix}{3}
        1 & 0 & 0 & 1/3 \\
        0 & 1 & 0 & 1/3 \\
        0 & 0 & 1 & 1/3 \\
        0 & 0 & 0 & 0
        
    \end{amatrix} \]
    This means there is only one solution to the system of equations:
    \[ \pi^*=\langle \frac{1}{3},\frac{1}{3},\frac{1}{3}\rangle \] \\
    But by definition of every stationary distribution $\pi$ must satisfy the requirements $\bP\pi=\pi$ and $\sum_{i=1}^n \pi_i=1$. Since $\pi^*$ is the only distribution which meats these requirements, then $\pi^*$ is the unique stationary distribution for $\bP$. 
    \begin{comment}
    \vspace{3mm}
    Approach 2: \\
    The matrix $\bP$ has eigenvalues $\lambda_1=1, \lambda_2=\frac{1}{5}, \lambda_3=0$. These have associated eigenvectors $v_1=\langle 1,1,1\rangle, v_2=\langle-1,0,1\rangle, v_3=\langle 1,\frac{-3}{2},1\rangle$.
    %  The eigenvalue approach seems problematic since it only gives long-term behavior (steady-state), not necessarily every stationary distribution.
    % This is because stationary distributions can be linear combinations of the bad eigenvectors like v_2=(-1,0,1).
    \end{comment}
\setcounter{enumi}{2}
\item
    Assume that $\pi$ is a stationary distribution, that $x$ leads to $y$,  and that $\pi(x)>0$ 
     The fact $x\to y$ means that $P^n(x,y)>0$ for some $n \in \mathbb{Z}^+$.  \\
     Using equation (3) from page 46, we get $ \sum_{a \in S}\pi(a)\bP^m(a,b)=\pi (b)$ for every $m>0,a,b \in S$. We know $\pi(x)>0$ by assumption and $\bP^n(x,y)> 0$ since $x\to y$. This means that $\pi(x)\bP^n(x,y)>0$ is positive. Since all other terms of the sum are nonnegative, this means that $\pi(y)>0$ is also positive. \qed
\end{enumerate}
\end{center}
\end{document}