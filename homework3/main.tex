\documentclass[14pt,letterpaper]{extarticle}
\usepackage[utf8]{inputenc}

% Set margins
% https://tex.stackexchange.com/a/327136
% this should replace use of the 'fullpage' package
\usepackage[
    includehead,
    margin=2.5cm,
    bmargin=2cm,
]{geometry}

\usepackage{amssymb}
\usepackage{amsthm}
\usepackage{verbatim} % This works better than comment for some reason
\usepackage{hyperref}
\usepackage{amsmath}
\usepackage{gensymb}
\usepackage{enumitem}
\usepackage{xfrac} % Fractions - https://tex.stackexchange.com/a/230973
\usepackage{xcolor}
% The magic latex drawing thing
\usepackage{tikz}
\usetikzlibrary{shapes}
\usepackage[en-US,showdow]{datetime2}

% Colored hyperlinks:
% https://www.overleaf.com/learn/latex/Hyperlinks
\hypersetup{
    colorlinks=true,
    urlcolor=blue
}

% Setup nice headings
% https://www.overleaf.com/learn/latex/Headers_and_footers
\usepackage{fancyhdr}
\usepackage{titling} % Save title for use in fancyhdr

% Shortcuts
\newcommand*{\bP}{\mathbf{P}}



% Use black tombstone for QED
\renewcommand{\qedsymbol}{$\blacksquare$}

% Important sets
\newcommand*{\C}{\mathbb{C}}
\newcommand*{\R}{\mathbb{R}}
\newcommand*{\Q}{\mathbb{Q}}
\newcommand*{\Z}{\mathbb{Z}}

%
% symbols
%

% name by meaning, not by character
\newcommand{\grad}{\nabla}


% New command for 'partial QED' using white tombstone
\newcommand*{\partialqed}{$\hfill \square$ \\}

\newcommand*{\todo}{{\color{red}\textbf{TODO:}} }

% Increased spaces spacing
\usepackage{setspace} \onehalfspacing

\title{Math 468 - Homework 3}
\author{Nicholas Schlabach (nickninja@arizona.edu)}
\DTMsavedate{titledate}{2026-02-11}
\date{\DTMUsedate{titledate}}

%% Copied from https://tex.stackexchange.com/a/2238
%% Modified by Techcable 9/19/21 to use '[...]' instead of '(...)'
\newenvironment{amatrix}[1]{%
  \left[\begin{array}{@{}*{#1}{c}|c@{}}
}{%
  \end{array}\right]
}

\begin{document}
\pagestyle{fancy}
\fancyhead[L]{Nicholas Schlabach}
\fancyhead[C]{\hspace{5mm}\Big(\thetitle{}\Big)}
\fancyhead[R]{{\DTMsetdatestyle{mmddyy}\DTMusedate{titledate}}, Page \thepage}
\fancyfoot{} % no footer
\maketitle
% maketitle automatically sets pagestyle=plain, override this
\thispagestyle{fancy}


\begin{center}
\begin{enumerate}
\setcounter{enumi}{1}
\item
    \begin{enumerate}
    \item
        This is not necessarily true for infinite markov chains. Consider the random walk on $\mathbb{Z}$ with probability $.9$ to go to the right and $.1$ to the left. All the states form a single communication class, as there is a nonzero probability of going between any two fixed states. It follows that the graph is both irreducible and closed. \\
        Despite this, the drift to the right means there are no recurring states.
    \item
        If $C$ is a finite closed communication class, then $C$ is recurrent.
    \item
        % Dr. Pickrell says every other step is valid
        % This interchange is not because [something dominated converage theorem]
        Interchanging the sum of limits with the limit of sums is not necessarily valid for infinite sums.
    \end{enumerate}
\item
    \begin{enumerate}
    \item
        % Expand $|p-q|^2=(p-q)^2=\big(p-(1-p)\big)^2=(2p-1)^2=4p^2-4p+1$. Using the fact 
        % If we can prove the squares are equal $|p-q|^2=1-4pq$ than this will imply that $|p-q|=(1-4pq)^{1/2}$: \\
        First I will prove that $|p-q|^2=-(1-4pq)$: \\
        \begin{tabular}{lll}
             & $|p-q|^2$ \\
             = & $(p-q)^2$ \\
             = & $p^2-2pq+q^2$  & [binomial theorem] \\
             = & $p^2-2p(1-p)+(1-p)^2$ \hspace{20mm} & [substitute $q=1-p$] \\
             = & $p^2-2p+2p^2+1^2-2p+p^2$ & [distribute \& binomial thm.] \\
             = & $4p^2-4p+1$ & [simplify] \\
             = & $4p(p-1)+1$ & [factor] \\
             = & $4pq+1$ & [definition of $q$] \\
             = & $-(1-4pq)$
        \end{tabular}
        Proving the squares are equal means the equation is equal up to sign. We know that $1-4pq\ge 0$ so then $-(1-4pq)\le 0$. 
    \item
        % TODO: Validity of E_x(N(y)) for recurrent states? The EV would be infinte in that case
        Part (a) of Thm. 1 tells us that $E_x(N(y))=\frac{p_{xy}}{1-p_{yy}}$. In this case $x=y$ and we get $E_0(N(0))=\frac{p_{yy}}{1-p_{yy}}$. Pattern matching with the value computed for $E_0(N(0)$ we get $p_yy=1-(1-4pq)^{1/2}$. Using the result of part (b) we get $\rho_{yy}=1-|p-q|$ as desired. \\
    \item
        If $p \ne \frac{1}{2}$, then $|p-q|\ne 0$ so $\rho_{yy}=1-|p-q|<1$. This is the definition of a transient state. \\
        If $p=\frac{1}{2}$ than $|p-q|=0$, so $\rho_{yy}=1-|p-q|=1$, which is the definition of a recurrent state.
    \end{enumerate}
\setcounter{enumi}{3}
\item
    First handle the case $0<p<1$: The states $\{1,\dots,d-1\}$ are transient and $\{0,d\}$ are recurrent. Fix a state $0<x<d$, because $T_0$ is absorbing, $P_x(T_0<T_d)$ is equivalent to $P_x(T_0<\infty)$ regardless of $0<x<d$. \\
    Professor says that $P_?(T_0<\infty)$ is called an "absorbption probability." The sum $P_1(T_0<\infty)=\sum_{r=0}^\infty {2r+1 \choose r}q^{r+1}p^r$ does not work because you can not choose $d+7$ moves right then $d+8$ moves left. You would encounter the absorbing state first! \\
    Instead, define the following function $\alpha(j)=P_j(T_0< \infty)$. Clearly $\alpha(0)=1$. \\
    Assuming we start at $0<j<d$, the event $\{T_0<\infty\}$ can be partitioned into $\{T_0<\infty,X_1=j-1\}, \{T_0<\infty,X_1=j+1\}$. \\
    Using the partition theorem we get the following:
    \[ \alpha(j)=P_{j-1}(T_0<\infty)q+P_{j-1}(T_0<\infty)p=\alpha(j-1)q+\alpha(j+1)p \]
    This is a second-order difference equation, analogous to a second-order differential equation $ay''(t)+by'(t)+cy(t)=0$. \\
    Guess: $\alpha(j)=\beta^j$ to get $\beta^j=q\beta^{j-1}+p\beta^{j+1}$. Simplifying gives $0=p\beta^2-\beta+q$.  This gives $\beta_{1,2}=\frac{1\pm \sqrt{1^2-4(pq)}}{2p}$ then we get $\alpha(j)=c_1\beta_1^j+c_2\beta_2^j$ where $c_1,c_2$ are determined by the boundary coefficients. \\
    We know that $\alpha(0)=1$ and $\alpha(d)=1$. Simplifying $1-4(pq)=1-4p(1-p)=1-4p+4p^2=(1-2p)^2$ then $\beta_{1,2}=\frac{1\pm(1-2p)}{2p}$. So $\beta_1=\frac{2-2p}{2p}=\frac{q}{p}, \beta_2=1$. So $\alpha$ takes the form $\alpha(j)=c_1(\frac{p}{q})^j+c_2$. Using the boundary conditions $\alpha(0)=1=c_1+c_2$ and $\alpha(d)=0=c_1(\frac{p}{q})^d+c_2$. This means that $c_2=1-c_1$ and $c_1=\big(1-(p/q)^d\big)^{-1}=1$. Plugging these coefficients back into $\alpha$ and then simplifying: 
    \[ \alpha(j)=\frac{(p/q)^j}{1-(p/q)^d}+1-\frac{1}{1-(p/q)^d}=\frac{(p/q)^j-(p/q)^d}{1-(p/q)^d}, \; 0<j<d \]
    But by definition, $\alpha(j)=P_j(T_0<\infty)$. Since state $d$ is absorbing, $P_j(T_0<\infty)=P_j(T_0<T_d)$ and we have the desired result.
\end{enumerate}
\end{center}
\end{document}