\documentclass[14pt,letterpaper]{extarticle}
\usepackage[utf8]{inputenc}

% Set margins
% https://tex.stackexchange.com/a/327136
% this should replace use of the 'fullpage' package
\usepackage[
    includehead,
    margin=2.5cm,
    bmargin=2cm,
]{geometry}

\usepackage{amssymb}
\usepackage{amsthm}
\usepackage{verbatim} % This works better than comment for some reason
\usepackage{hyperref}
\usepackage{amsmath}
\usepackage{gensymb}
\usepackage{enumitem}
\usepackage{xfrac} % Fractions - https://tex.stackexchange.com/a/230973
\usepackage{xcolor}
% The magic latex drawing thing
\usepackage{tikz}
\usetikzlibrary{shapes}
\usepackage[en-US,showdow]{datetime2}

% Colored hyperlinks:
% https://www.overleaf.com/learn/latex/Hyperlinks
\hypersetup{
    colorlinks=true,
    urlcolor=blue
}

% Setup nice headings
% https://www.overleaf.com/learn/latex/Headers_and_footers
\usepackage{fancyhdr}
\usepackage{titling} % Save title for use in fancyhdr

% Shortcuts
\newcommand*{\bP}{\mathbf{P}}



% Use black tombstone for QED
\renewcommand{\qedsymbol}{$\blacksquare$}

% Important sets
\newcommand*{\C}{\mathbb{C}}
\newcommand*{\R}{\mathbb{R}}
\newcommand*{\Q}{\mathbb{Q}}
\newcommand*{\Z}{\mathbb{Z}}

%
% symbols
%

% name by meaning, not by character
\newcommand{\grad}{\nabla}


% New command for 'partial QED' using white tombstone
\newcommand*{\partialqed}{$\hfill \square$ \\}

\newcommand*{\todo}{{\color{red}\textbf{TODO:}} }

% Increased spaces spacing
\usepackage{setspace} \onehalfspacing

\title{Math 468 - Homework 1}
\author{Nicholas Schlabach (nickninja@arizona.edu)}
\DTMsavedate{titledate}{2026-02-04}
\date{\DTMUsedate{titledate}}

%% Copied from https://tex.stackexchange.com/a/2238
%% Modified by Techcable 9/19/21 to use '[...]' instead of '(...)'
\newenvironment{amatrix}[1]{%
  \left[\begin{array}{@{}*{#1}{c}|c@{}}
}{%
  \end{array}\right]
}

\begin{document}
\pagestyle{fancy}
\fancyhead[L]{Nicholas Schlabach}
\fancyhead[C]{\hspace{5mm}\Big(\thetitle{}\Big)}
\fancyhead[R]{{\DTMsetdatestyle{mmddyy}\DTMusedate{titledate}}, Page \thepage}
\fancyfoot{} % no footer
\maketitle
% maketitle automatically sets pagestyle=plain, override this
\thispagestyle{fancy}


\begin{center}
\begin{enumerate}
\item
    \begin{enumerate}
    \item
        This case has the following transition matrix:
        \[ \bP=\begin{bmatrix}
            1-p & p \\
            0 & 1 
        \end{bmatrix} \]
        Since $\bP(1,0)=q=0$, then $\rho_{10}=0$ and state $s_1$ does not lead to state $s_0$. This means $s_0$ and $s_1$ do not communicate and so fall into separate communication classes, making this a reducible markov chain. Since $\bP(1,1)=1$ then $s_1$ is absorbing and  $\rho_{11}=1$. This means that $s_1$ is also recurrent. Since $P(0,1)>0\implies \rho_{01}>1$ and $1$ does not lead back to $0$, then state $s_0$ is transient. The communication class $\{s_1\}$ is closed but $\{s_0\}$ is not closed.
    \item
        In the case $p=q=0$,  the transition matrix $\bP$ is the identity matrix. Both states $s_0, s_1$ are absorbing meaning they are both recurrent. Neither state leads to the other so there are two communication classes and the system is irreducible. Both communication classes are closed. \\
        In the case $p=q=1$, both states communicate $s_0 \leftrightarrow s_1$ and so there is a single cloased communication class containing both states. In fact, because $P(0,1)=1=P(1,0)$ then both states are recurrent.
    \end{enumerate}

\end{enumerate}
\end{center}
\end{document}